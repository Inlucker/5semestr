\StructuredChapter{Введение}

Видео и фото тесно вплелись в нашу жизнь. Почти каждый
мобильный телефон оснащен камерой. Почти каждая камера умеет записывать видео. Повсеместно распространилась 3D-графика.
С развитием возможностей усиливается потребность в “дешевом”
построении 3D-сцен. Самый очевидный из таких методов -- стереозрение -- получение трехмерной картины мира по видеоряду
или нескольким изображениям.

Есть и другие возможные применения. В киноиндустрии -- для
создания спецэффектов и 3D-фильмов. В военном деле -- для измерения расстояний. Как правило, камера
расположена сбоку от монитора, где отображается лицо собеседника. Таким образом, другой участник смотрит куда-то в сторону.
Построение трехмерного изображения лица позволяет синтезировать новое изображение, на котором собеседник смотрит прямо в
глаза, что усиливает эффект присутствия.

Совмещение изображений, позволяет человеку получить информацию о расстоянии до объектов по их расхождениям (disparity).
Эта идея может быть использована в алгоритмах обработки изображений. Хотя, конечно же, зрение человека активное (то есть параметры оптической системы настраиваются под изображение) -- глаза вращаются в глазницах, меняется фокусное расстояние, как
правило, построение системы активного зрения сложнее и дороже,
чем зрения пассивного.

Система пассивного стереозрения, как правило, включает в себя 2 камеры и большинство алгоритмов решают общую задачу сопоставления 2-х изображений.

Существуют различные классификации алгоритмов сопоставления 2-х. Один из вариантов такой классификации представлен
в \cite{3}. Все алгоритмы делятся на локальные (в которых расхождение вычисляется в каждой точке на основе “похожести” окна вокруг этой точки и окна вокруг точек на другом изображении) и
глобальные (основанные на минимизации функционала энергии --
мы находим расхождение сразу для всех точек). Глобальные в свою
очередь делятся по способу минимизации энергии. Как правило, это
динамическое программирование \cite{4,5,6,7} или нахождение минимального разреза графа \cite{8,9}. Алгоритмы разреза графа называют также двумерными. Они дают довольно точные результаты, но имеют
меньшую производительность. Алгоритмы, обрабатывающие строки изображений независимо друг от друга \cite{4,5}, называют одномерными. Они работают быстрее, но подвержены эффекту гребенки, с
которым борются с помощью разных ухищрений. Нечто среднее по
производительности и качеству из себя представляют алгоритмы
оптимизации на поддереве графа, построенного на пикселях изображения.

Многие алгоритмы построены на модели случайных полей Маркова (MRF). Основное предположение в такой модели: расхождение в любой точке зависит только от расхождений соседних точек
(как правило, считают, что их 4, хотя можно соседними считать
и 8 точек, оставаясь в рамках модели). Минимизация энергии в
них производится на основе разрезов графа или распространения
доверия \cite{10}.

...

\textbf{Цель работы} – анализ методов применяемых при построении объёмного изображения по стереопаре.

\textbf{Задачи работы:}
\begin{itemize}
	\item описать термины предметной области и обозначить проблему;
	\item провести обзор существующих программных решений в области стереограмметрии;
	\item выбрать критерии для их оценки и сравнить;
	\item выбрать наиболее предпочтительный.
\end{itemize}
