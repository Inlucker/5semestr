\addcontentsline{toc}{chapter}{Список использованных источников}
\UnnamedStructuredChapter{Список использованных источников}
\bibliographystyle{utf8gost705u}
\begin{thebibliography}{1}
	\bibitem{3}
	Scharstein D., Szeliski R. A taxonomy and evaluation of dense
	two-frame stereo correspondence algorithms // Int. Journal of
	Computer Vision 47. April-June 2002. С. 7–42.
	
	\bibitem{4}
	Bobick A. F., Intille S. S. Large occlusion stereo // Int. Journal
	of Computer Vision. 33(3). 1999. С. 181–200.
	
	\bibitem{5}
	Ohta Y., Kanade T. Stereo by intra- and inter- scanline search
	using dynamic programming // IEEE TPAMI. 7(2). 1985.
	С. 139–154.
	
	\bibitem{6}
	Veksler O. Stereo correspondence by dynamic programming on a
	tree // Proc. CVPR. Vol. 2. 2005. С. 384–390.
	
	\bibitem{7}
	Naveed I. R., Huijun Di, GuangYou Xu Refine stereo
	correspondence using bayesian network and dynamic programming
	on a color based minimal span tree // ACIVS. 2006. С. 610–619.
	
	\bibitem{8}
	Kolmogorov V., Zabih R. Computing visual correspondence with
	occlusions using graph cuts // ICCV. Vol. 2. 2001. С. 508–515.
	
	\bibitem{9}
	Boykov Y., Veksler O., Zabih R. Fast approximate energy
	minimization via graph cuts // IEEE TPAMI 23(11). 2001.
	С. 1222–1239.
	
	\bibitem{10}
	Sun J., Shum H., Zheng .N Stereo matching using belief
	propagation // In ECCV. 2002. С. 510–524.
	
	%\bibitem{RANSAC}
	%Derpanis K.G. Overview of the RANSAC Algorithm. // 2010 С. 1-2.
	
	\bibitem{RANSAC}
	Fischler M. A. and Bolles R. C. Random sample consensus: a paradigm for model fitting with applications to image analysis and automated cartography. // Commun. ACM 24. 1981. С. 381-395
	
	\bibitem{Sift}
	Lowe D. G. Object recognition from local scale invariant features // Proceedings of the Seventh IEEE International Conference on Computer Vision 2. 1999. С. 1150-1157
	
	\bibitem{Sift2}
	Lowe D. G. Distinctive Image Features from Scale-Invariant Keypoints. // International Journal of Computer Vision. 2004. С. 91-110
	
	\bibitem{DSI}
	Тупицын И. В. Реконструкция трехмерной модели объекта на основе стереопары при решении задач 3D-моделирования. // Сибирский аэрокосмический журнал. 2011. С. 88-92
\end{thebibliography}
