\sloppy

% --------------------------------------------------------- %

\usepackage{wrapfig}
\usepackage{float}
\usepackage{microtype}

\usepackage{setspace}

\usepackage{lipsum}

\newcommand{\code}[1]{\texttt{#1}}

\makeatletter
\newcommand{\vhrulefill}[1]
{
	\leavevmode\leaders\hrule\@height#1\hfill \kern\z@
}
\makeatother

% --------------------------------------------------------- %

% Настройки стиля ГОСТ 7-32
% Для начала определяем, хотим мы или нет, чтобы рисунки и таблицы нумеровались в пределах раздела, или нам нужна сквозная нумерация.
\EqInChapter % формулы будут нумероваться в пределах раздела
\TableInChapter % таблицы будут нумероваться в пределах раздела
\PicInChapter % рисунки будут нумероваться в пределах раздела

% Добавляем гипертекстовое оглавление в PDF
\usepackage[
bookmarks=true, colorlinks=true, unicode=true,
urlcolor=black,linkcolor=black, anchorcolor=black,
citecolor=black, menucolor=black, filecolor=black,
]{hyperref}

\AfterHyperrefFix

\usepackage{microtype}% полезный пакет для микротипографии, увы под xelatex мало чего умеет, но под pdflatex хорошо улучшает читаемость

% Тире могут быть невидимы в Adobe Reader
\ifInvisibleDashes
\MakeDashesBold
\fi

\usepackage{graphicx}   % Пакет для включения рисунков

% С такими оно полями оно работает по-умолчанию:
% \RequirePackage[left=20mm,right=10mm,top=20mm,bottom=20mm,headsep=0pt,includefoot]{geometry}
% Если вас тошнит от поля в 10мм --- увеличивайте до 20-ти, ну и про переплёт не забывайте:
\geometry{right=20mm}
\geometry{left=30mm}
\geometry{bottom=20mm}
\geometry{ignorefoot}% считать от нижней границы текста


% Пакет Tikz
\usepackage{tikz}
\usetikzlibrary{arrows,positioning,shadows}

% Произвольная нумерация списков.
\usepackage{enumerate}

% ячейки в несколько строчек
\usepackage{multirow}

% itemize внутри tabular
\usepackage{paralist,array}

%\setlength{\parskip}{1ex plus0.5ex minus0.5ex} % разрыв между абзацами
\setlength{\parskip}{1ex} % разрыв между абзацами
\usepackage{blindtext}

% Центрирование подписей к плавающим окружениям
%\usepackage[justification=centering]{caption}

\usepackage{newfloat}
\DeclareFloatingEnvironment[
placement={!ht},
name=Equation
]{eqndescNoIndent}
\edef\fixEqndesc{\noexpand\setlength{\noexpand\parindent}{\the\parindent}\noexpand\setlength{\noexpand\parskip}{\the\parskip}}
\newenvironment{eqndesc}[1][!ht]{%
    \begin{eqndescNoIndent}[#1]%
\fixEqndesc%
}
{\end{eqndescNoIndent}}

% ---------------------------------------------------------------- %

\usepackage{listings}
\usepackage{listingsutf8}
\lstset{
	basicstyle=\small,
	keywordstyle=\color{blue},
	stringstyle=\color{red},
	commentstyle=\color{gray},
	numbers=left,
	stepnumber=1,
	numberstyle=\tiny,
	numbersep=5pt,
	language=CIL,
	tabsize=2,
	frame=single,
	breaklines=true,
	breakatwhitespace=true
}

\newcommand{\imgwc}[4]
{
	\begin{figure}[#1]
		\center{\includegraphics[width=#2]{inc/img/#3}}
		\caption{#4}
		\label{img:#3}
	\end{figure}
}
\newcommand{\imghc}[4]
{
	\begin{figure}[#1]
		\center{\includegraphics[height=#2]{inc/img/#3}}
		\caption{#4}
		\label{img:#3}
	\end{figure}
}
\newcommand{\imgsc}[4]
{
	\begin{figure}[#1]
		\center{\includegraphics[scale=#2]{inc/img/#3}}
		\caption{#4}
		\label{img:#3}
	\end{figure}
}

\titleformat{\section}[block]{\normalfont\bfseries}{\thesection}{14pt}{}
\titleformat{\subsection}[block]{\normalfont\bfseries}{\thesubsection}{14pt}{}
\titleformat{\subsubsection}[block]{\normalfont\bfseries}{\thesubsubsection}{14pt}{}

\titleformat{\chapter}[block]
	{\normalfont\fontsize{14}{14}\bfseries}
	{\thechapter}
	{1em}{}
\titlespacing{\chapter}{\parindent}{0mm}{5mm}

\newcommand{\UnnamedStructuredChapter}[1]
{
	\titleformat{\chapter}[block]
  	{\centering\normalfont\large\bfseries}
  	{\centering\thechapter}
  	{14pt}{\centering\large}
		\chapter*{\MakeUppercase{#1}}
	\titleformat{\chapter}[block]
  	{\normalfont\large\bfseries}
  	{\thechapter}
  	{14pt}{\large}
}

\newcommand{\StructuredChapter}[1]
{
	\titleformat{\chapter}[block]
  	{\centering\normalfont\large\bfseries}
  	{\centering\thechapter}
  	{14pt}{\centering\large}
		\chapter*{\MakeUppercase{#1}}
	\addcontentsline{toc}{chapter}{#1}
	\titleformat{\chapter}[block]
  	{\normalfont\large\bfseries}
  	{\thechapter}
  	{14pt}{\large}
}

\usepackage{caption}
\DeclareCaptionLabelSeparator{bar}{\space---\space}
\captionsetup{labelsep=bar}
\addto\captionsrussian
{
	\def\figurename{Рисунок}
	\def\contentsname{\normalsize СОДЕРЖАНИЕ}
}

\newcommand\labelitemi{\small$\bullet$}
\newcommand\labelitemii{\small$\bullet$}
\newcommand\labelitemiii{\small$\bullet$}

\lstset{
  literate={а}{{\selectfont\char224}}1
           {б}{{\selectfont\char225}}1
           {в}{{\selectfont\char226}}1
           {г}{{\selectfont\char227}}1
           {д}{{\selectfont\char228}}1
           {е}{{\selectfont\char229}}1
           {ё}{{\"e}}1
           {ж}{{\selectfont\char230}}1
           {з}{{\selectfont\char231}}1
           {и}{{\selectfont\char232}}1
           {й}{{\selectfont\char233}}1
           {к}{{\selectfont\char234}}1
           {л}{{\selectfont\char235}}1
           {м}{{\selectfont\char236}}1
           {н}{{\selectfont\char237}}1
           {о}{{\selectfont\char238}}1
           {п}{{\selectfont\char239}}1
           {р}{{\selectfont\char240}}1
           {с}{{\selectfont\char241}}1
           {т}{{\selectfont\char242}}1
           {у}{{\selectfont\char243}}1
           {ф}{{\selectfont\char244}}1
           {х}{{\selectfont\char245}}1
           {ц}{{\selectfont\char246}}1
           {ч}{{\selectfont\char247}}1
           {ш}{{\selectfont\char248}}1
           {щ}{{\selectfont\char249}}1
           {ъ}{{\selectfont\char250}}1
           {ы}{{\selectfont\char251}}1
           {ь}{{\selectfont\char252}}1
           {э}{{\selectfont\char253}}1
           {ю}{{\selectfont\char254}}1
           {я}{{\selectfont\char255}}1
           {А}{{\selectfont\char192}}1
           {Б}{{\selectfont\char193}}1
           {В}{{\selectfont\char194}}1
           {Г}{{\selectfont\char195}}1
           {Д}{{\selectfont\char196}}1
           {Е}{{\selectfont\char197}}1
           {Ё}{{\"E}}1
           {Ж}{{\selectfont\char198}}1
           {З}{{\selectfont\char199}}1
           {И}{{\selectfont\char200}}1
           {Й}{{\selectfont\char201}}1
           {К}{{\selectfont\char202}}1
           {Л}{{\selectfont\char203}}1
           {М}{{\selectfont\char204}}1
           {Н}{{\selectfont\char205}}1
           {О}{{\selectfont\char206}}1
           {П}{{\selectfont\char207}}1
           {Р}{{\selectfont\char208}}1
           {С}{{\selectfont\char209}}1
           {Т}{{\selectfont\char210}}1
           {У}{{\selectfont\char211}}1
           {Ф}{{\selectfont\char212}}1
           {Х}{{\selectfont\char213}}1
           {Ц}{{\selectfont\char214}}1
           {Ч}{{\selectfont\char215}}1
           {Ш}{{\selectfont\char216}}1
           {Щ}{{\selectfont\char217}}1
           {Ъ}{{\selectfont\char218}}1
           {Ы}{{\selectfont\char219}}1
           {Ь}{{\selectfont\char220}}1
           {Э}{{\selectfont\char221}}1
           {Ю}{{\selectfont\char222}}1
           {Я}{{\selectfont\char223}}1
}

\newcommand{\specialcell}[2][c]
{
  \begin{tabular}[#1]{@{}c@{}}#2\end{tabular}
}