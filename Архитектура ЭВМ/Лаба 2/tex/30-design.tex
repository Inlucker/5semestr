\chapter{Задание 1}
\label{cha:design}

\begin{lstlisting}[label=code:source_1, caption=Листинг исходной программы]
        .section .text
		.globl _start;
		len = 9 #Размер массива
		enroll = 2 #Количество обрабатываемых элементов за одну итерацию
		elem_sz = 4 #Размер одного элемента массива

_start:
		la x1, _x
		addi x20, x1, elem_sz*(len)#Адрес элемента, следующего за последним
		lw x31, 0(x1)
		addi x1, x1, elem_sz*1
lp:
		lw x2, 0(x1)
		lw x3, 4(x1) #!
		bltu x2, x31, lt1
		add x31, x0, x2
lt1:    bltu x3, x31, lt2
		add x31, x0, x3
lt2:
		add x1, x1, elem_sz*enroll
		bne x1, x20, lp
lp2: j lp2

		.section .data
_x:     .4byte 0x1
		.4byte 0x2
		.4byte 0x3
		.4byte 0x4
		.4byte 0x5
		.4byte 0x6
		.4byte 0x7
		.4byte 0x31
		.4byte 0x9
\end{lstlisting}

\newpage
\begin{lstlisting}[label=code:source_21, caption=Деассемблированный листинг исходной программы]
Disassembly of section .text:

80000000 <_start>:
80000000:       00000097                auipc   x1,0x0
80000004:       03808093                addi    x1,x1,56 # 80000038 <_x>
80000008:       02808a13                addi    x20,x1,40
8000000c:       0000af83                lw      x31,0(x1)
80000010:       00408093                addi    x1,x1,4

80000014 <lp>:
80000014:       0000a103                lw      x2,0(x1)
80000018:       0040a183                lw      x3,4(x1)
8000001c:       01f16463                bltu    x2,x31,80000024 <lt1>
80000020:       00200fb3                add     x31,x0,x2

80000024 <lt1>:
80000024:       01f1e463                bltu    x3,x31,8000002c <lt2>
80000028:       00300fb3                add     x31,x0,x3

8000002c <lt2>:
8000002c:       00808093                addi    x1,x1,8
80000030:       ff4092e3                bne     x1,x20,80000014 <lp>

80000034 <lp2>:
80000034:       0000006f                jal     x0,80000034 <lp2>

Disassembly of section .data:

80000038 <_x>:
80000038:       0001                    c.addi  x0,0
8000003a:       0000                    unimp
8000003c:       0002                    0x2
8000003e:       0000                    unimp
80000040:       00000003                lb      x0,0(x0) # 0 <enroll-0x2>
80000044:       0004                    c.addi4spn      x9,x2,0
80000046:       0000                    unimp
80000048:       0005                    c.addi  x0,1
8000004a:       0000                    unimp
8000004c:       0006                    0x6
8000004e:       0000                    unimp
80000050:       00000007                0x7
80000054:       0008                    c.addi4spn      x10,x2,0
80000056:       0000                    unimp
80000058:       0009                    c.addi  x0,2
\end{lstlisting}

Ниже, в листинге \ref{code:source_31} приведен псевдокод на языке C эквивалентной программы.

\begin{lstlisting}[label=code:source_31, caption=Псевдокод на языке C эквивалентной программы]
#define len 9
#define enroll 2
#define elem_sz 4

void _start()
{
	int _x[] = {1,2,3,4,5,6,7,31,9};
	int x2, x3, x4, x5, x31;
	
	int *x1 = _x;
	int *x20 = x1 + len;
	x31 = x1[0];
	x1 += 1;
	
	do
	{
		x2 = x1[0];
		x3 = x1[1];
		if (x2>=x31)
		x31 = x2;
		if (x3>=x31)
		x31 = x3;
		x1 += enroll;
	} while(x1 != x20);
	
	x1 += enroll;
	
	while(1){}
}
\end{lstlisting}

В результате выполнения данного участка кода в регистр x31 будет находится максимальный элемент массива, то есть значение 0x31.
