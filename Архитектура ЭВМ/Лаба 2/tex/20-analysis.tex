\chapter{Аналитическая часть}
\label{cha:analysis}

\section{Архитектура набора команд RV32I}

RISC-V является открытым современным набором команд, который может использоваться для построения как микроконтроллеров, так и высокопроизводительных микропроцессоров. В связи с такой широкой областью применения в систему команд введена вариативность. Таким образом, термин RISC-V фактически является названием для семейства различных систем команд, которые строятся вокруг базового набора команд, путем внесения в него различных расширений.

В данной работе исследуется набор команд RV32I, который включает в себя основные команды 32-битной целочисленной арифметики кроме умножения и деления. В рамках данного набора команд мы не будем рассматривать системные команды, связанные с таймерами, системными регистрами, управлением привилегиями, прерываниями и исключениями.

Набор команд RV32I предполагает использование 32 регистров общего назначения x0-x31 размером в 32 бита каждый и регистр pc, хранящего адрес следующей команды. Все регистры общего назначения равноправны, в любой команде могут использоваться любые из регистров. Регистр pc не может использоваться в командах.

Архитектура RV32I предполагает плоское линейное 32-х битное адресное пространство. Минимальной адресуемой единицей информации является 1 байт. Используется порядок байтов от младшего к старшему (Little Endian), то есть, младший байт 32-х битного слова находится по младшему адресу (по смещению 0). Отсутствует разделение на адресные пространства команд, данных и ввода-вывода. Распределение областей памяти между различными устройствами (ОЗУ, ПЗУ, устройства ввода-вывода) определяется реализацией.

\section{Микроархитектура}

В лабораторной работе рассматривается система, состоящая из вычислительного ядра Taiga и локальной памяти, реализованной с помощью блочной памяти ПЛИС. Команды и данные находятся в едином адресном пространстве. Дешифратор адресов настроен таким образом, что блок памяти ПЛИС отображается в адресное пространство RISC-V с адреса 0x80000000. Память ПЛИС имеет фиксированную задержку доступа в 1 такт, в связи с чем отпадает необходимость в кеш-памяти. 

Taiga является конвейерным микропроцессором с элементами суперскалярности. При конвейерной организации микропроцессора различные команды одновременно проходят различные стадии своей обработки. Конвейер Taiga насчитывает 4 стадии. В скобках приведены сокращенные обозначения стадий.

\begin{enumerate}
	\item Выборка(F) --- cтадия, на которой команда извлекается из ПК. Выполняется в блоке выборки.

	\item Диспетчеризация (ID) --- стадия, на которой происходит запись команды в очередь команд для декодирования. Выполняется в блоке управления метаданными.

	\item Декодирование и планирование на выполнение (D) --- стадия на которой происходит определение типа и полей команды и определение вычислительного блока, способного ее исполнить. Выполняется в блоке декодирования и планирования на выполнение.

	\item Выполнение (AL, M1..M3, в зависимости от исполнительного блока) --- стадия, на которой команда передается в блок выполнения.	
\end{enumerate}

"Ширина" конвейера Taiga равна 1 для всех стадий, кроме стадии выполнения. В лучшем случае, каждая стадия конвейера выполняется за один такт.

В состав рассматриваемой конфигурации Taiga входит 3 блока выполнения команд:
Арифметико-логическое устройство (АЛУ), блок доступа к памяти (LSU) и блок ветвлений.
АЛУ и блок ветвлений выполняют команды за 1 такт, LSU — минимум за 3. 

Ниже, на рисунке \ref{img:taiga_pipeline} приведена структурная схема ядра Taiga.

\imgsc{H}{0.40}{taiga_pipeline}{Обобщенная структурная схема ядра Taiga}