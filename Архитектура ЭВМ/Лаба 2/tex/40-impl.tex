\chapter{Задания 2, 3, 4}
\label{cha:impl}

Мой вариант 17, следовательно в соответствии с заданием $2$ выполняем поиск такта, в котором выполняется выборка команды с адрессом $0x80000020$, номер итерации: 2-я. на рисунке \ref{img:FetchCompleteMy} приведена временная диаграмма, поясняющая этапы выборки и диспетчеризации.

\imgsc{H}{0.6}{FetchCompleteMy}{Временнуя диаграмма выполнения стадий выборки и диспетчеризации команды}

В соответствии с заданием 3 выполняем поиск такта, в котором выполняется декодирование и планирование команды с адрессом $0x8000002$с, номер итерации: 2-я. На рисунке \ref{img:DecodeAndIssue} приведена временная диаграмма, поясняющая этап декодирования и планирования.

\imgsc{H}{0.6}{DecodeAndIssue}{Временнуя диаграмма выполнения стадии декодирования и планирования на выполнение команды}

В соответствии с заданием 4 выполняем поиск такта, в котором выполняется исполнение команды с адрессом $0x80000018$, номер итерации: 2-я. На рисунке \ref{img:Execute} приведена временная диаграмма, поясняющая этап выполнения команды.

\imgsc{H}{0.6}{Execute}{Диаграмма, соответствующая этапу выполнения}

\chapter{Задание 5}
На рисунках \ref{img:fetchANDdispatch5}-\ref{img:execute5} изображены временные диаграммы сигналов, соответствующих всем стадиям выполнения команды, обозначенной в тексте программы символом \#!.

\imgsc{H}{0.6}{fetchANDdispatch5}{Диаграмма, соответствующая этапам выборки и диспетчеризации команды, обозначенной в тексте программы символом \#!}

\imgsc{H}{0.6}{decodeANDissue5}{Диаграмма, соответствующая этапам декодирования и планирования команды, обозначенной в тексте программы символом \#!}

\newpage
\imgsc{H}{0.6}{execute5}{Диаграмма, соответствующая этапу выполнения команды, обозначенной в тексте программы символом \#!}

Результат выполнения программы занесен в регистр x31. Как и предполагалось, в нем хранится значение 0x31 на момент окончания работы программы (рисунок \ref{img:rez}).

\imgsc{H}{0.6}{rez}{Результат выполнения программы}

Следующие задание: анализируя диаграмму заполнить трассу выполнения программы.

Ниже, на рисунке \ref{img:pipeline} приведена трасса работы программы.

\imgsc{H}{0.35}{pipeline}{Трасса работы программы}