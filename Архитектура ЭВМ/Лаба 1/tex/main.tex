\DefaultChapter{Введение}
\textbf{Цель работы:} Изучение основ построения микропроцессорных систем на ПЛИС. В ходе работы студенты ознакомятся с принципами построения систем на кристалле (СНК) на основе ПЛИС, получат навыки проектирования СНК в САПР Altera Quartus II, выполнят проектирование и верификацию системы с использованием отладочного комплекта Altera DE1Board.

Система на кристалле (SoC, СНК) --- это функционально законченная электронная вычислительная система, состоящая из одного или нескольких микропроцессорных модулей, а также системных и периферийных котроллеров, выполненная на одном кристалле. Такая
тесная интеграция компонентов системы позволяет достичь высокого быстродействия при
построении специализированных ЭВМ.

\newpage

\section{Структура проектируемой СНК}

Перед тем, как приступить к практической части, было описано предназначение основных блоков СНК:

1) Микропроцессорное ядро Nios II/e выполняет функции управления системой.

2) Внутренняя оперативная память СНК, используемая для хранения программы управления и данных.

3) Системная шина Avalon обеспечивает связность всех компонентов системы. 

4) Блок синхронизации и сброса обеспечивает обработку входных сигналов сброса и синхронизации и распределение их в системе.

5) Блок идентификации версии проекта обеспечивает хранение и выдачу уникального идентификатора версии, который используется программой управления при инициализации системы.	

6) Контроллер UART обеспечивает прием и передачу информации по интерфейсу RS232.

Ниже, на рисунке \ref{img:func_scheme} приведена структура проектируемой системы на кристалле.

\imgsc{H}{0.5}{func_scheme}{Функциональная схема разрабатываемой системы на кристалле}

Для данной работы процесс верификация системы подразумевает написание программы, запрашивающей SystemID и выводящей его в отладочную консоль. В данном случае SystemID указывался при добавлении в проект блока идентификации (См. рисунок \ref{img:sysid_qsys}). 

\imgsc{H}{0.5}{sysid_qsys}{Задание SystemID}

\section{Практическая часть}

После ознакомления со структурой была сконструирована СНК с помощью средства проектирования систем на кристалле Altera Qsys, созданный в соответствии с методическими указаниями. Ниже, на рисунке \ref{img:nios.qsys} приведен результат проектирования, а также таблица распредения адресов модулей СНК.

\imgsc{H}{0.45}{nios.qsys}{Модуль в QSYS и таблица распределения адресов}

\imgsc{H}{0.35}{Pin planner}{Pin planner}

\imgsc{H}{0.35}{Programmer}{Programmer}

\newpage
Для верификации системы была написана программа в среде разработки Nios II Software Build Tools for Eclipse для отображения значения SystemID.

После этого к компьютеру со средой Quartus II была подключена плата Altera Cyclon II FPGA Started Board. После настройки соединения была запущена на ней программа.

Ниже, на рисунках \ref{img:code}-\ref{img:work_result} приведены код и результаты работы программы.

\imgsc{H}{0.7}{code}{Код программы}

\imgsc{H}{1}{work_result}{Пример работы программы}

\DefaultChapter{Заключение}

В итоге, было спроектировано СНК в САПР Altera Quartus II и выполнено его "тестирование"\ с использованием созданной в рамках лабораторной работы программы.

Таким образом все поставленные задачи решены, цель работы достигнута.