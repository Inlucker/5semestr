\setcounter{chapter}{1}

\section*{Задание 1}
\addcontentsline{toc}{section}{Задание 1}

\textbf{Задание}: Ознакомиться с возможностями программы PCLAB в Разделе 2 методических указаний. Запустить программу PCLAB 1.0. Изучить идентификационную информацию на вкладке «Идентификация процессора». 

Описание PCLAB: Программа PCLAB предназначена для исследования производительности x86 совместимых ЭВМ c IA32 архитектурой, работающих под управлением операционной системы Windows (версий 95 и старше). Исследование организации ЭВМ заключается в проведении ряда экспериментов, направленных на построение зависимостей времени обработки критических участков кода от изменяемых параметров. 

Набор реализуемых программой экспериментов позволяет исследовать особенности построения современных подсистем памяти ЭВМ и процессорных устройств, выявить конструктивные параметры конкретных моделей ЭВМ. Процесс сбора и анализа экспериментальных данных в PCLAB основан на процедуре профилировки критического кода, т.е. в измерении времени его обработки центральным процессорным устройством.

\section*{Задание 2}
\addcontentsline{toc}{section}{Задание 2}

\textbf{Задание}: На основании идентификационной информации о микропроцессоре ЭВМ, используемой при проведении лабораторной работы, определить следующие параметры: размер линейки кэш-памяти верхнего уровня и объем физической памяти. Результаты занести в отчет.

Размер линейки кэша: 64 байт;
Объем физической памяти: 4 Гб.

\section*{Задание 3}
\addcontentsline{toc}{section}{Задание 3}

\textbf{Задание}: Ознакомиться с описанием эксперимента «Исследование расслоения динамической памяти» на вкладке «Описание эксперимента». Провести эксперимент. По результатам эксперимента определить: количество банков динамической памяти; размер одной страницы динамической памяти; количество страниц в динамической памяти. Сделать выводы о использованном способе наращивания динамической памяти. Результаты занести в отчет.

\textbf{Цель эксперимента}: определение способа трансляции физического адреса, используемого при обращении к динамической памяти.

\textbf{Исходные данные}:
\begin{itemize}
	\item Размер линейки кэша: 64 байт;
	\item Объем физической памяти: 4 Гбайт.
\end{itemize}

В таблице \ref{tab_3} приведены настраиваемые параметры.
\begin{table}[H]
	\begin{center}
		\caption{Настраиваемые параметры}
		\label{tab_3}
		\begin{tabular}{|c|c|c|c|}
		\hline
		Эксперимент & № & Значение & Описание 	\\
		\hline
		\hline
		\multirow{3}*{1} & 1 & 32 Кбайт & \specialcell{Максимальное расстояния \\ между читаемыми блоками} 	\\
		\cline{2-4} & 2 & 128 байт &\specialcell{ Шаг увеличения расстояния между \\ читаемыми 4-х байтовыми ячейками}	\\
		\cline{2-4} & 3 & 1 Мбайт & Размер массива		\\
		\hline
		\multirow{3}*{2} & 1 & 32 Кбайт & \specialcell{Максимальное расстояния \\ между читаемыми блоками} 	\\
		\cline{2-4} & 2 & 64 байт &\specialcell{ Шаг увеличения расстояния между \\ читаемыми 4-х байтовыми ячейками}	\\
		\cline{2-4} & 3 & 1 Мбайт & Размер массива		\\
		\hline
		\end{tabular}
	\end{center}
\end{table}

Ниже, на рисунках \ref{img:task_01} - \ref{img:task_11} приведены зависимости времени обращения к памяти от расстояния между читаемыми блоками данных.

\imgsc{H}{0.4}{task_01}{Результаты исследования расслоения динамической памяти (часть 1)}
\imgsc{H}{0.4}{task_21}{Результаты исследования расслоения динамической памяти (часть 2)}
\imgsc{H}{0.4}{task_11}{Результаты исследования расслоения динамической памяти (часть 3)}

Ниже, на рисунке \ref{img:bank_memory} приведена блочно-циклическая схема расслоения памяти, помогающая более ясно представить суть эксперимента.

\imgsc{H}{0.85}{bank_memory}{Блочно-циклическая схема расслоения памяти}

\textbf{Результаты эксперимента}:
\begin{itemize}
	\item количество банков динамической памяти: $B = T1/P = 1024/128 = 8$;
	\item размер одной страницы динамической памяти: $PS = T2/B = 4096/8 = 512\ byte$;
	\item количество страниц в динамической памяти: $C = V/(PS*B*P) = \frac{2^{32}}{2^{19}} = 8192\ byte$.
\end{itemize}

Примечание:
\begin{itemize}
	\item B --- количество банок памяти;
	\item P --- объем данных, являющийся минимальной порцией обмена кэш-памяти верхнего уровня с оперативной памятью;
	\item PS --- размер страницы DRAM памяти;
	\item T1 --- размер блока в одной банке памяти;
	\item T2 --- размер страницы одной банки памяти;
	\item V --- объем физического пространства ОП;
	\item C --- количество страниц в ОП.
\end{itemize}

\textbf{Выводы}:
\begin{enumerate}
	\item Память расслоена, доступ к ней с разным временем доступа, в зависимости от размера запрашиваемого блока.
	\item Экспериметально были получены значения количества банков ОП, размер страница банка, размер страницы DRAM.
	\item Данные следует размещать так, чтобы они укладывали в одну страницу.
	\item Данные следует выравнивать по размеру линейки кэша.
	\item Данные следует обрабатывать так, чтобы минимизировать количество последовательных обращений в одну банку ОП.
\end{enumerate} 

\section*{Задание 4}
\addcontentsline{toc}{section}{Задание 4}

\textbf{Задание}: Ознакомиться с описанием эксперимента «Сравнение эффективности ссылочных и векторных структур данных». Провести эксперимент. По результатам эксперимента определить: отношение времени работы алгоритма, использующего зависимые данные, ко времени обработки аналогичного алгоритма обработки независимых данных. Сделать выводы об эффективности ссылочных и векторных структур данных и способах ее повышения. Результаты занести в отчет.

\textbf{Цель эксперимента}: оценить влияние зависимости команд по данным на эффективность вычислений.

В таблице \ref{tab_4} приведены настраиваемые параметры.
\begin{table}[H]
	\begin{center}
		\caption{Настраиваемые параметры}
		\label{tab_4}
		\begin{tabular}{|c|c|c|c|}
		\hline
		№ & Значение & Описание 	\\
		\hline
		\hline
		1 & 1 Мбайт & \specialcell{Количество элементов в списке} 	\\
		\hline
		2 & 32 Кбайт & \specialcell{Максимальная фрагментации списка}		\\
		\hline
		3 & 1 Кбайт & \specialcell{Шаг увеличения фрагментации}	\\
		\hline
		\end{tabular}
	\end{center}
\end{table}

Ниже, на рисунке \ref{img:task_02} приведена зависимость времени выполнения поиска минимального значения для массива и односвязного списка.

\imgsc{H}{0.41}{task_02}{Сравнение эффективности ссылочных и векторных структур данных}

Из полученного графика видна проблема сематического разрыва. Следует использовать структуры данных с учётом технологического фактора. Для машины «лучше» использовать массив т.к. она плохо работает со списками.

\textbf{Результаты эксперимента}. Односвязный список обрабатывался в \textbf{19,741378} раз дольше. \\

\textbf{Выводы}:
\begin{enumerate}
	\item Связанные данные следует организовывать так, чтобы при работе программы они были как можно ближе друг к другу расположены в ОП.
	\item Использование структур данных, помогающих ясней представить и быстрее решить задачу, может приводить к значительному снижению производительности системы (Таким образом, в эксперименте наблюдали следствие сематического разрыва)
\end{enumerate} 

\section*{Задание 5}
\addcontentsline{toc}{section}{Задание 5}

\textbf{Задание}: Для ЭВМ, используемой при проведении лабораторной работы определить следующие параметры: степень ассоциативности и размер TLB данных. Ознакомиться с описанием и провести эксперимент «Исследование эффективности программной предвыборки». По результатам эксперимента определить: отношение времени последовательной обработки блока данных ко времени обработки блока с применением предвыборки; время и количество тактов первого обращения к странице данных. Сделать выводы об эффективности предвыборки и способах ее повышения. Результаты занести в отчет.

\textbf{Цель эксперимента}: выявить способы ускорения вычислений благодаря применению предвыборки данных.

\textbf{Исходные данные}: 
\begin{itemize}
	\item степень ассоциативности TLB данных: 4 ячейки;
	\item размер TLB данных: 128 групп.
\end{itemize}

В таблице \ref{tab_5} приведены настраиваемые параметры.
\begin{table}[H]
	\begin{center}
		\caption{Настраиваемые параметры}
		\label{tab_5}
		\begin{tabular}{|c|c|c|c|}
		\hline
		№ & Значение & Описание 	\\
		\hline
		\hline
		1 & 512 байт & \specialcell{Шаг увеличения расстояния \\ между читаемыми данными} \\
		\hline
		2 & 128 Кбайт & \specialcell{Размер массива}		\\
		\hline
		\end{tabular}
	\end{center}
\end{table}

Ниже, на рисунке \ref{img:task_03} приведены зависимости времени обращения к памяти от расстояния между читаемыми блоками данных. Красный график показывает время или количество тактов работы алгоритма без предвыборки. Зеленый график показывает время или количество тактов работы алгоритма с использованием предвыборки.

\imgsc{H}{0.4}{task_03}{Исследование эффективности программной предвыборки}

Время обращения к первому элементу в таблицы в 20 раз больше т.к. мы не знаем где это страница находится в памяти, мы имеем только логический адрес, а нужен физический. Поэтому логично использовать предвыборку. В данном примере ускорение почти в 2 раза. 

\textbf{Результаты эксперимента}:
\begin{itemize}
	\item обработка без загрузки таблицы страниц в TLB производилась в \textbf{1,8456621} раз дольше;
	\item время первого обращения к странице данных: \textbf{15`000} тактов.
\end{itemize}

\textbf{Вывод}: Для исключения задержек, связанных с получением физического адреса начала страницы, имеет смысл предварительно загрузить страницы в TLB перед работой с большими массивами данных.

\section*{Задание 6}
\addcontentsline{toc}{section}{Задание 6}

\textbf{Задание}: Ознакомиться с описанием и провести эксперимент «Исследование способов эффективного чтения оперативной памяти». По результатам эксперимента определить: отношение времени обработки блока памяти неоптимизированной структуры ко времени обработки блока структуры, обеспечивающей эффективную загрузку и параллельную обработку данных. Сделать выводы о способах повышения эффективности чтения оперативной памяти.

\textbf{Цель эксперимента}: исследование возможности ускорения вычислений благодаря
использованию структур данных, оптимизирующих механизм чтения оперативной памяти.

\textbf{Исходные данные}: 
\begin{itemize}
	\item адресное расстояние между банками памяти: 128 байт;
	\item размер буфера чтения:	4 КБайт.
\end{itemize}

В таблице \ref{tab_6} приведены настраиваемые параметры.
\begin{table}[H]
	\begin{center}
		\caption{Настраиваемые параметры}
		\label{tab_6}
		\begin{tabular}{|c|c|c|c|}
		\hline
		№ & Значение & Описание 	\\
		\hline
		\hline
		1 & 1 Мбайт & \specialcell{Размер массива} \\
		\hline
		2 & 128 ед. & \specialcell{Количество потоков данных}		\\
		\hline
		\end{tabular}
	\end{center}
\end{table}

Ниже, на рисунке \ref{img:task_04} приведены зависимости времени чтения данных от количества одновременно обрабатываемых массивов для неоптимизированной структуры (красный график) и структуры, обеспечивающей эффективную загрузку и параллельную обработку данных (зеленый график).

\imgsc{H}{0.4}{task_04}{Исследование способов эффективного чтения оперативной памяти}

\textbf{Результаты эксперимента}:
\begin{itemize}
	\item отношение времени обработки блока памяти неоптимизированной структуры ко времени обработки блока структуры, обеспечивающей эффективную загрузку и параллельную обработку данных: \textbf{1.65}.
\end{itemize}

\textbf{Вывод}: 
\begin{itemize}	
	\item упорядочив данные определённым образом, можно ускорить приложение (мы группируем данные, которые используем вместе.);
	\item следует переупорядочивать данные, выравнивая их по размеру кэш-линии, тем самым исключая несвоевременную передачу данных;
	\item следует размещать данные как можно ближе друг к другу: стараться, по возможности, не обращаться к диспетчеру кучи за памятью, помнить и использовать на приктике особенности  выравнивания данных (например в С-структурах);
\end{itemize}

\section*{Задание 7}
\addcontentsline{toc}{section}{Задание 7}

\textbf{Задание}: Для ЭВМ, используемой при проведении лабораторной работы определить следующие параметры: размер банка кэш-памяти данных первого и второго уровня, степень ассоциативности кэш-памяти первого и второго уровня, размер линейки кэш-памяти первого и второго уровня. Ознакомиться с описанием и провести эксперимент «Исследование конфликтов в кэш-памяти». По результатам эксперимента определить: отношение времени обработки массива с конфликтами в кэш-памяти ко времени обработки массива без конфликтов. Сделать выводы о способах устранения конфликтов в кэш-памяти. 

\textbf{Цель эксперимента}: исследование влияния конфликтов кэш-памяти на эффективность
вычислений.

\textbf{Исходные данные}:
\begin{itemize}
	\item размер банка кэш-памяти данных первого и второго уровня: 32 КБайт;
	\item степень ассоциативности кэш-памяти первого и второго уровня: 8 ячеек;
	\item размер линейки кэш памяти первого и второго уровня:	64 байт.
\end{itemize}

В таблице \ref{tab_7} приведены настраиваемые параметры.
\begin{table}[H]
	\begin{center}
		\caption{Настраиваемые параметры}
		\label{tab_7}
		\begin{tabular}{|c|c|c|c|}
		\hline
		№ & Значение & Описание 	\\
		\hline
		\hline
		1 & 128 Кбайт & \specialcell{Размер банка кэш-памяти} \\
		\hline
		2 & 128 байт & \specialcell{Размер линейки кэш-памяти}		\\
		\hline
		3 & 32 ед. & \specialcell{Количество читаемых линеек}		\\
		\hline
		\end{tabular}
	\end{center}
\end{table}

Ниже, на рисунке \ref{img:task_05} приведены зависимости времени обращения к памяти от расстояния между читаемыми блоками данных.

\imgsc{H}{0.4}{task_05}{Исследование конфликтов в кэш-памяти}

\textbf{Результаты эксперимента}: отношение времени обработки массива с конфликтами в кэш-памяти ко времени обработки массива без конфликтов = \textbf{5,9547263}.

Ниже, на рисунке \ref{img:cache} приведена кэш-память с четырехканальным частично-ассоциативным отображением, помогающая понять суть эксперимента.

\imgsc{H}{0.75}{cache}{Кэш-память с четырехканальным частично-ассоциативным отображением}

\textbf{Вывод}:
\begin{itemize}
	\item кэш память ускоряет процессор примерно в 6 раз;
	\item данные следует выравнивать по адресам, кратным размеру линейки кэша;
	\item данные следует обрабатывать так, чтобы уменьшить количество последовательных обращений к блокам памяти, соответствующих одному набору (модулю) (рисунок \ref{img:cache}).
\end{itemize}

\section*{Задание 8}
\addcontentsline{toc}{section}{Задание 8}

\textbf{Задание}: Ознакомиться с описанием и провести эксперимент «Исследование алгоритмов сортировки». По результатам эксперимента определить: отношение времени сортировки массивов алгоритмом QuickSort ко времени сортировки алгоритмом Counting Radix, а также ко времени сортировки Counting-Radix алгоритмом, оптимизированным под 8-процессорную вычислительную систему. Сделать выводы о наиболее эффективном алгоритме сортировки.

\textbf{Цель эксперимента}: исследование способов эффективного использования памяти и
выявление наиболее эффективных алгоритмов сортировки, применимых в вычислительных
системах.

В таблице \ref{tab_8} приведены настраиваемые параметры.
\begin{table}[H]
	\begin{center}
		\caption{Настраиваемые параметры}
		\label{tab_8}
		\begin{tabular}{|c|c|c|c|}
		\hline
		№ & Значение & Описание 	\\
		\hline
		\hline
		1 & 1 Мбайт & \specialcell{Количество 64-х разрядных \\ элементов массивов} \\
		\hline
		2 & 8 Кбайт & \specialcell{Шаг увеличения размера массива}		\\
		\hline
		\end{tabular}
	\end{center}
\end{table}

Ниже, на рисунке \ref{img:task_06} приведены зависимости времени сортировки (алгоритмы Quick Sort, Radix-Counting Sort, Оптимизированный Radix-Counting Sort) от размера исходного массива.

\imgsc{H}{0.4}{task_06}{Исследование алгоритмов сортировки}

\textbf{Примечание}: Фиолетовый график показывает время или количество тактов работы алгоритма QuickSort. Красный график показывает время или количество тактовработы неоптимизированного алгоритма Radix-Counting. Зеленый график показывает время или количество тактов работы оптимизированного под 8-процессорную вычислительную систему алгоритма Radix-Counting.

\textbf{Результаты эксперимента}:
\begin{itemize}
	\item отношение времени сортировки массива алгоритмом QuickSort ко времени сортировки алгоритмом Radix-Counting Sort: 	\textbf{1.8379134}; 
	\item отношение времени сортировки массива алгоритмом QuickSort ко времени сортировки
Radix-Counting Sort, оптимизированной под 8-процессорную вычислительную систему: \textbf{2.0982673}.
\end{itemize}

\textbf{Выводы}: 
\begin{itemize}
	\item существует алгоритм поразрядной сортировки с сложности меньше чем линейной вычислительной сложности O(n/log(n));
	\item следует выбирать алгоритмы на основе типа входных данных, которые позволяют решить задачу наиболее эффективно.
\end{itemize}