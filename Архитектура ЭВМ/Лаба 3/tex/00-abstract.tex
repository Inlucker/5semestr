% Также можно использовать \Referat, как в оригинале
\begin{abstract}

    Отчет содержит \pageref{LastPage}\,стр.%
    \ifnum \totfig >0
    , \totfig~рис.%
    \fi
    \ifnum \tottab >0
    , \tottab~табл.%
    \fi
    %
    \ifnum \totbib >0
    , \totbib~источн.%
    \fi
    %
    \ifnum \totapp >0
    , \totapp~прил.%
    \else
    .%
    \fi


    Это пример каркаса расчётно-пояснительной записки, желательный к использованию в РПЗ проекта по курсу РСОИ
    \nocite{*}.

    Данный опус, как и более новые версии этого документа, можно взять по адресу (\url{https://github.com/latex-g7-32/latex-g7-32}).

    Текст в документе носит совершенно абстрактный характер.

\end{abstract}

%%% Local Variables: 
%%% mode: latex
%%% TeX-master: "rpz"
%%% End: 
