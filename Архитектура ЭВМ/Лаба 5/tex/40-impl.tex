\chapter{Задания 2, 3, 4}
\label{cha:impl}

В соответствии с заданием $2$ выполняем поиск такта, в котором выполняется выборка команды с адрессом $0x80000034$. Таким образом, в такте 15 команда добавляется в таблицу команд $pc\_table$ и в конце этого же такта ей присваивается $id = 5$. Выборка не произошла в 14 такте из-за того, что был выставлен сигнал $pc\_id\_assigment = 0$, что свидельствует о том, что очередь команд в менеджере метаинформации заполнена. В 16 такте заканчивается этап декодирования и код операции добавляется в таблицу $instruction\_table$.

Ниже, на рисунке \ref{img:task_02} приведена временная диаграмма, поясняющая этапы выборки и диспетчеризации.

\imgsc{H}{0.45}{task_02}{Диаграмма, соответствующая этапам выборки и диспетчеризации}

В соответствии с заданием 3 выполняем поиск такта, в котором выполняется декодирование и планирование команды с $pc\_id = 5$ (с адрессом $0x80000014$). В конце 10 такта виден результат декодирования (выполнение команды делегируется блоку доступа к памяти).

Ниже, на рисунке \ref{img:task_03} приведена временная диаграмма, поясняющая этап декодирования и планирования.

\imgsc{H}{0.40}{task_03}{Диаграмма, соответствующая этапам декодирования и планирования}

В соответствии с заданием 4 выполняем поиск такта, в котором выполняется исполнение команды с $pc_id = 2$ (команда была выбрана из адресса $0x80000028$).

Ниже, на рисунке \ref{img:task_04} приведена временная диаграмма, поясняющая этап выполнения команды.

\imgsc{H}{0.40}{task_04}{Диаграмма, соответствующая этапу выполнения}

Результат выполнения программы занесен в регистр x31. Как и предполагалось, в нем хранится значение 0x25 на момент окончания работы программы (рисунок \ref{img:task_041}).

\imgsc{H}{0.75}{task_041}{Результат выполнения программы}