\setcounter{chapter}{1}

\section{Исходные тексты программ}

Ниже в листинге \ref{code:code_1} приведен изначальный код программы (ядра для ускорителя) в соответствии с моим вариантом.

\begin{lstlisting}[label=code:code_1, caption=Код исходной программы]
	extern "C" {
		void var014(int* c, const int* a, const int* b, const int len) {
			int tmpA = 0;
			int tmpB = 0;
			for (int i = 0; i < len; i++) {
				tmpA += a[i] * i;
				tmpB += b[i] * i;
			}
			for (int i = 0; i < len; i+=2) {
				c[i] = tmpA;
				c[i+1] = tmpB;
			}
		}
	}
\end{lstlisting}

Ниже в листингах \ref{code:code_2} - \ref{code:code_4} приведен код программ, в которых использовались директивы, указывающие компилятору оптимизировать программу.

\begin{lstlisting}[label=code:code_2, caption=Конвейерное исполнение]
	extern "C" {
		void var014(int* c, const int* a, const int* b, const int len) {
			int tmpA = 0;
			int tmpB = 0;
			for (int i = 0; i < len; i++) {
				#pragma HLS PIPELINE
				tmpA += a[i] * i;
				tmpB += b[i] * i;
			}
			for (int i = 0; i < len; i+=2) {
				#pragma HLS PIPELINE
				c[i] = tmpA;
				c[i+1] = tmpB;
			}
		}
	}
\end{lstlisting}


\newpage
\begin{lstlisting}[label=code:code_3, caption=Развернутый цикл]
	extern "C" {
		void var014(int* c, const int* a, const int* b, const int len) {
			int tmpA = 0;
			int tmpB = 0;
			for (int i = 0; i < len; i++) {
				#pragma HLS UNROLL  factor=2
				tmpA += a[i] * i;
				tmpB += b[i] * i;
			}
			for (int i = 0; i < len; i+=2) {
				#pragma HLS UNROLL  factor=2
				c[i] = tmpA;
				c[i+1] = tmpB;
			}
		}
	}
\end{lstlisting}


\begin{lstlisting}[label=code:code_4, caption=Развернутый цикл и конвейерное исполнение]
	extern "C" {
		void var014(int* c, const int* a, const int* b, const int len) {
			int tmpA = 0;
			int tmpB = 0;
			for (int i = 0; i < len; i++) {
				#pragma HLS UNROLL  factor=2
				#pragma HLS PIPELINE
				tmpA += a[i] * i;
				tmpB += b[i] * i;
			}
			for (int i = 0; i < len; i+=2) {
				#pragma HLS UNROLL  factor=2
				#pragma HLS PIPELINE
				c[i] = tmpA;
				c[i+1] = tmpB;
			}
		}
	}
\end{lstlisting}

\newpage
\section{Результаты работы}

Ниже на рисунке \ref{img:sw-console} приведены результаты эмуляции выполнения ядра на центральном процессоре.

\imgsc{H}{0.65}{sw-console}{Результаты программной эмуляции}

Ниже на рисунках \ref{img:assistant_1} - \ref{img:assistant_2} приведена копия экрана с открытым Assistant view.

\imgsc{H}{0.45}{assistant_1}{Копия экрана с открытым Assistant view}

\imgsc{H}{0.45}{assistant_2}{Копия экрана с открытым Assistant view (продолжение)}

%\imgsc{H}{0.5}{emulation_рw_result}{Результаты программной эмуляции}

Ниже на рисунке \ref{img:hw} приведены результаты работы приложения в режиме Emulation-HW.

\imgsc{H}{0.35}{hw}{Результаты работы приложения в режиме Emulation-HW}

\imgsc{H}{0.75}{hw-console}{Сравнительная таблица с временем выполнения}

Ниже на рисунках \ref{img:hardware-console} - \ref{img:platform_diagram} приведены результаты синтеза программы для ускорителя.

\imgsc{H}{0.65}{hardware-console}{Сравнительная таблица с временем выполнения}

\imgsc{H}{0.35}{system_diagram}{Диаграмма системы}

\imgsc{H}{0.35}{platform_diagram}{Диаграмма платформы}

% Сделать примечание, что я заместо деления использовал сдвиг право

\imgsc{H}{0.45}{hls_no_pragmas}{Информация о компонентах для реализации без оптимизации}

\imgsc{H}{0.45}{hls_pipelined}{Информация о компонентах для реализации с конвейеризацией}

\imgsc{H}{0.45}{hls_pipe_unroll}{Информация о компонентах для реализации с развертыванием циклов}

\imgsc{H}{0.45}{hls_unrolled}{Информация о компонентах для реализации с конвейеризацией и развертыванием циклов}
