\newpage
\section{Заключение}
При конвейерной обработке цикл может начинать последующие
итерации цикла менее чем за три такта. Конвейерная обработка цикла
приводит к разворачиванию любых циклов, вложенных внутрь конвей­
ерного цикла. Если внутри цикла существуют зависимости по данным,
может оказаться невозможным достичь запуска новой итерации в каждом
такте, и результатом может быть больший интервал инициации.

Разворачивание циклов является общепризнанным механизмом
снижения времени выполнения циклов. Все развернутые итераций цикла
будут выполняться параллельно, поэтому для реализации такого варианта
оптимизации потребуется больший объем программируемых логических
ресурсов. В результате компилятор может столкнуться с проблемами,
связанными с таким большим количеством ресурсов, и с проблемами
емкости, которые замедляют процесс компиляции ядра.

Оба варианта оптимизации помогли уменьшить количество тактов
на выполнение цикла. Комбинирование двух вариантов оптимизации
также привело к оптимизации программы.

\newpage
\section{Ответы на контрольные вопросы}

\textbf{1. Преимущества и недостатки аппаратных ускорителей на ПЛИС по сравнению с CPU и графическими ускорителями?}

В отличие от CPU и GPU, программируемые устройства представляют собой полностью настраиваемую архитектуру, которую разработчик может использовать для размещения вычислительных блоков с требуемой функциональностью. Возможность настроить аппаратное обеспечение под специализированную задачу позволяет достичь высокой производительности. Также ПЛИС позволяет достичь лучшего показателя энергоэффективности

 Графические процессоры масштабируют производительность за счет большого количества ядер и использования параллелизма SIMD/SIMT (Рисунок 1).  В таком случает, высокий уровень производительности достигается за счет создания длинных конвейеров обработки данных, а не за счет увеличения количества вычислительных единиц. Понимание этих преимуществ является необходимым условием для разработки вычислительных устройств и достижения наилучшего уровня ускорения.

\textbf{2. Основные способы оптимизации циклических конструкций ЯВУ, реализуемых в виде аппаратных ускорителей?}

\begin{enumerate}
	\item Конвейеризация. Позволяет повысить пропускная способность, за
	счет увеличения времени синтеза программы и требований к ресурса­
	ми ПЛИС. При конвейеризации вместо комплексного преобразования
	входных данных в одной сложной схеме используются последовательные
	простые операции, каждая из которых выполняется в своем цифровом
	узле, а промежуточные результаты запоминаются в триггерах. Это упро­
	щение преобразований позволяет уменьшить число последовательных
	ячеек от триггера до триггера, повысив, таким образом, частоту. Для
	указания компилятору о необходимости конвейеризировать циклическую
	обработку используется директива PIPELINE;
	\item Разворачивание циклов. Для указания компилятору о необходи­
	мости развернуть цикл используется директива UNROLL: его итерации
	начинают выполняться параллельно на собственном наборе оборудования.
	Количество тактов на выполнение всего цикла уменьшается за счет роста
	размера схемы.
\end{enumerate}

\textbf{3. Назовите этапы работы программной части ускорителя в хост системе?}
\begin{enumerate} 
	\item Вычисление размера массива; 
	\item Объявление и инициализация исходных массивов;
	\item Получение списка устройств и инициализация контекста;
	\item Создание контекста и очередей команд к устройствам;
	\item Получение необходимой информации об устройстве;
	\item Создание программного объекта opencL и загрузка программы в
	двоичном формате на ускоритель;
	\item Выделение памяти под буферы устройства;
	\item Для запуска каждой реализации
	\begin{enumerate} 
		\item Устанавливаем необходимые для тестирования значения;
		\item Копируем содержимое буферов в DDR память ускорительной
		карты;
		\item Заупскаем задачу на исполнение и ждем готовности по преры­
		ванию;
		\item Читаем метки времени исполнения задачи;
		\item Читаем данные из DDR памяти устройства в буфер результатов;
	\end{enumerate}
\end{enumerate}


\textbf{4. В чем заключается процесс отладки для вариантов сборки Emulation-SW, Emulation-HW и Hardware?}

При отладке в режиме программной эмуляцим код ядра компили­
руется для работы на ЦПУ хост-системы. Этот вариант сборки служит
для верификации совместного исполнения кода хост-системы и кода ядра, для выявления синтаксических ошибок, выполнения отладки на уровне
исходного кода ядра, понимания или проверки поведения системы.

Для отладки в режиме аппаратной эмуляции код ядра компили­
руется в аппаратную модель (RTL), которая запускается в специальном
симуляторе на ЦПУ. Этот вариант сборки и запуска занимает больше вре­
мени, но обеспечивает подробное и точное представление активности ядра.
Данный вариант сборки полезен для тестирования функциональности
ускорителя и получения начальных оценок производительности.

Для отладки в режиме аппаратного обеспечения (Hardware) код
ядра компилируется в аппаратную модель (RTL), а затем реализуется на
FPGA. В результате формируется двоичный файл xclbin, который будет
работать на реальной FPGA.



\textbf{5. Какие инструменты и средства анализа результатов синтеза возможно использовать в Vitis HLS для оптимизации ускорителей?}

\begin{itemize}
	\item отладчик, имеющий графический интерфейс;
	\item использование конструкций, указывающих компилятору путь оптимизации (прагмы и директивы $(set\_directive\_*)$);
	\item средство анализа Vivado IDE, позволяющее в том числе оценивать
	время и затраты после синтеза или размещения, выполнять симуляцию
	выполнения программы на ускорителе. Также Vivado IDE позволяет после
	высокоуровнего синтеза оптимизировать проекты на уровне межрегистро­
	вых передач.
\end{itemize}
