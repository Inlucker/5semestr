\chapter{Задание 1}
\label{cha:design}

\begin{lstlisting}[label=code:source_1, caption=Листинг исходной программы]
        .section .text
        .globl _start;
        len = 8 
        enroll = 4
		elem_sz = 4
_start:
        la x1, _x
        addi x20, x1, elem_sz*len #Адрес последнего элемента
lp:
        lw x2, 0(x1)
        add x31, x31, x2 #!
        lw x3, 4(x1)
        add x31, x31, x3
        lw x4, 8(x1)
        add x31, x31, x4
        lw x5, 12(x1)
        add x31, x31, x5
        addi x1, x1, elem_sz*enroll
        bne x1, x20, lp
        addi x31, x31, 1
lp2: j lp2

        .section .data
_x:     .4byte 0x1
        .4byte 0x2
        .4byte 0x3
        .4byte 0x4
        .4byte 0x5
        .4byte 0x6
        .4byte 0x7
        .4byte 0x8
\end{lstlisting}

\begin{lstlisting}[label=code:source_21, caption=Деассемблированный листинг исходной программы]
Disassembly of section .text:

80000000 <_start>:
80000000:       00000097                auipc   x1,0x0
80000004:       03c08093                addi    x1,x1,60 # 8000003c <_x>
80000008:       02008a13                addi    x20,x1,32
\end{lstlisting}
\begin{lstlisting}[label=code:source_22, caption=Продолжение листинга \ref{code:source_21}]
8000000c <lp>:
8000000c:       0000a103                lw      x2,0(x1)
80000010:       002f8fb3                add     x31,x31,x2
80000014:       0040a183                lw      x3,4(x1)
80000018:       003f8fb3                add     x31,x31,x3
8000001c:       0080a203                lw      x4,8(x1)
80000020:       004f8fb3                add     x31,x31,x4
80000024:       00c0a283                lw      x5,12(x1)
80000028:       005f8fb3                add     x31,x31,x5
8000002c:       01008093                addi    x1,x1,16
80000030:       fd409ee3                bne     x1,x20,8000000c <lp>
80000034:       001f8f93                addi    x31,x31,1

80000038 <lp2>:
80000038:       0000006f                jal     x0,80000038 <lp2>
\end{lstlisting}

Ниже, в листинге \ref{code:source_31} приведен псевдокод на языке C эквивалентной программы.

\begin{lstlisting}[label=code:source_31, caption=Псевдокод на языке C эквивалентной программы]
#define len 8
#define enroll 4
#define elem_sz 4

int _x[] = {1,2,3,4,5,6,7,8};
int x2, x3, x4, x5, x31;

void _start()
{
	int *x1 = _x;
    int x20 = len;

    do
    {
      x2 = x1[0];
      x31 += x2;
      x3 = x1[1];
      x31 += x3;
      x4 = x1[2];
      x31 += x4;
      x5 = x1[3];
      x31 += x5;
      x1 += enroll;
\end{lstlisting}
\begin{lstlisting}[label=code:source_32, caption=Продолжение листинга \ref{code:source_31}]
    } while(x1 != x20);
    x31++;

    while(1){}
}
\end{lstlisting}

В результате выполнения данного участка кода в регистр x31 будет находится инкрементированная  сумма элементов массива, то есть значение 0x25.
